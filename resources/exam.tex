% Author: Carsten Gips <carsten.gips@fh-bielefeld.de>
% Copyright: (c) 2016-2018 Carsten Gips
% License: MIT


%------------------------------------- Packages ----------------
\usepackage[absolute]{textpos}
\usepackage{colortbl}
\usepackage{wasysym}    % \Square
\usepackage{ifoddpage}
%------------------------------------- Packages ----------------

%------------------------------------- Settings ---------------------
\extrawidth{.5in}
%------------------------------------- Settings ---------------------

%------------------------------------- CMDs ---------------------
\newcommand{\hsfont}    {\fontfamily{phv}\fontseries{m}\fontshape{n}\selectfont}
\newcommand{\hsheadfont}{\fontfamily{phv}\fontseries{b}\fontshape{n}\selectfont}

%% `[...]{.answer}` will be translated by the lua filter into `\x{...}`
\newcommand{\x}[1]{\ifprintanswers{\color{red}\bfseries#1\xspace}\fi}

%% `[...]{.punkte}` will be printed in the lower right coner of a solution box
\newcommand{\p}[1]{\vfill\ttfamily\bfseries#1\xspace}

\newcommand{\Fortsetzung}{\begin{textblock*}{54mm}(80mm,276mm)\textsl{\textbf{Fortsetzung nächste Seite}}\end{textblock*}}
\newcommand{\Leerseite}{\newpage\centerline{\textsl{\textbf{Leerseite}}}\newpage}
%------------------------------------- CMDs ---------------------

%------------------------------------- Listings ---------------------
%% settings for pandoc option `--listings`
\usepackage{listings}  
\lstset{basicstyle=\footnotesize\ttfamily\mdseries, xleftmargin=\bigskipamount, keywordstyle=\bfseries\color{dkblue}, identifierstyle=\ttfamily, commentstyle=\bfseries\color{gray}\textsl, stringstyle=\color{magenta}\upshape, emphstyle=\color{red}, emphstyle={[2]\color{blue}}, texcl=false, showspaces=false, showstringspaces=false, numbers=left, numberstyle=\footnotesize, breaklines=true, tabsize=4, backgroundcolor=\color{listinggray}, frame=shadowbox}
%------------------------------------- Listings ---------------------

%------------------------------------- Tables (left column gray background) --------------------------------
\newenvironment{streifenenv}
{
    \smallskip
    \begin{tabular}{>{\columncolor{headcolor}}p{1pt}p{0.86\textwidth}}
        &
        \begin{minipage}{0.86\textwidth}
}
{
        \end{minipage}
    \end{tabular}
    \smallskip
}

% use more space for table rows in exams to allow students to fill out blank cells
% use less space in solutions: tables often fills a page, but we need to add grading information to the table
\newcommand{\rowstretch}{\ifprintanswers \renewcommand{\arraystretch}{0.6} \else \renewcommand{\arraystretch}{2.0} \fi}
% use a little more space for table rows in multiple choice questions
\newcommand{\mcstretch}{\renewcommand{\arraystretch}{1.2}}
% reset the table row settings
\newcommand{\rowrelax}{\renewcommand{\arraystretch}{1.0}}
% definitions for multiple choice questions
\newcommand{\antwort}{\ifprintanswers \ensuremath{\blacksquare} \else \ensuremath{\Box} \fi}
\newcommand{\wahr}{\antwort & \ensuremath{\Box} & }
\newcommand{\falsch}{\ensuremath{\Box} & \antwort & }
%------------------------------------- Tables (left column gray background) --------------------------------

%------------------------------------- Answers --------------------------------
\CorrectChoiceEmphasis{\color{red}\bfseries}
\checkboxchar{\ensuremath{\Box}}
\checkedchar{\ensuremath{\blacksquare}}
\shadedsolutions
\definecolor{SolutionColor}{rgb}{0.9,0.8,0.8}
%------------------------------------- Answers --------------------------------

%------------------------------------- Grading Table --------------------------------
\chqword{\textbf{Aufgabe}}
\chpword{\textbf{Punkte}}
\chbpword{\textbf{Bonus}}
\chsword{\raisebox{-1mm}[6mm][4mm]{\textbf{Erreicht}}}
\chtword{\raisebox{-1mm}[6mm][4mm]{\textcolor{headcolor}{\LARGE\ensuremath{\;\;\;\pmb{\Sigma}\;\;\;}}}}
%------------------------------------- Grading Table --------------------------------

%------------------------------------- Custom Title Page ----------------
\renewcommand{\maketitle}{} % we use "coverpage" from the exam package instead
\renewcommand{\tableofcontents}{} % we have to use --toc to compile the exam twice, but we do not want really a toc
%------------------------------------- Custom Title Page ----------------

%------------------------------------- Custom Header ----------------
\pagestyle{headandfoot}     %% from exam.cls
\headrule
\ifprintanswers
\header{}{\textcolor{dkred}{\textbf{\Huge LÖSUNG}}}{}
\else
\header{Name:}{Matrikelnummer:}{\thepage/\numpages}
\fi
\footrule
%% footer-definition for single-sided printing (empty backside)
%\footer{$LVKURZ$}{Klausur $if(PART)$$PART$ \xspace$endif$ $NR$}{Seite \thepage\ von \numpages}
%% footer-definition for double-sided printing
\lfoot{$LVKURZ$}
\cfoot{Klausur $if(PART)$$PART$ \xspace$endif$ $NR$}
\rfoot{\framebox[1.5cm]{\rule{0pt}{.7cm}}}
%------------------------------------- Custom Header ---------------------

%------------------------------------- Questions --------------------------------
\providecommand{\theMyQuestionTitle}{}
\providecommand{\theMyBonusQuestionTitle}{}
\providecommand{\myQuestion}[2][0]{\vskip11pt\renewcommand{\theMyQuestionTitle}{#2}\ifnum0#1>0 \question[#1]\else \question \fi {\ }\vskip5pt}
\providecommand{\myBonusQuestion}[2][0]{\vskip11pt\renewcommand{\theMyBonusQuestionTitle}{#2}\bonusquestion[#1]{\ }\vskip5pt}
%\boxedpoints    % no effect w/ qformat
\qformat{\textbf{\textcolor{headcolor}{\Large $if(PART)$$PART$.$endif$\thequestion: \theMyQuestionTitle}\hfill {\Large\texttt{\thepoints}}}}  % add only question points
\bonusqformat{\textbf{\textcolor{headcolor}{\Large $if(PART)$$PART$.$endif$\thequestion: \theMyBonusQuestionTitle}\hfill {\Large\texttt{\thepoints}}}}  % add only question points
\pointpoints{Punkt}{Punkte}
\bonuspointpoints{Bonuspunkt}{Bonuspunkte}
\totalformat{Punkte für Frage \thequestion: \totalpoints}
\bonustotalformat{Bonuspunkte für Frage \thequestion: \totalbonuspoints}
\renewcommand{\solutiontitle}{\noindent\textbf{Lösung:}\enspace}
%------------------------------------- Questions --------------------------------



